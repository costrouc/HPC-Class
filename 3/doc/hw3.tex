\documentclass[11pt]{article}
\usepackage{fullpage}

\usepackage{algorithm}
\usepackage{algpseudocode}

\begin{document}

\title{Assignment 3}
\author{Chris Ostrouchov}
\date{}
\maketitle

\section{Introduction}

The last part of homework 3 was to implement Cholesky QR factorization. Below you can see the pseudocode for the following Algorithm \ref{alg:chol_qr} implemented in \textbf{src/}. As requested from part 2 comment are provided with the algorithm. The most expensive operations in code consist of the matrix multiplications, svd, and qr factorizations each consisting of $O(mn^{2})$ operations. As we will see in homework 4 these operations can be speed up using GPUs. 

Each of following tests were implemented on a 4 core intel i5 3570k processor. Table \ref{tab:i5_3570k} details the specifications of the processor. 

\begin{algorithm}
  \caption{Cholesky QR Factorization}
  \label{alg:chol_qr}
  \begin{algorithmic}[1]
    \Procedure{Cholesky QR}{$A$}
    
    \State $Q \gets A$
    \State $G \gets Q^{T}Q$ \Comment{\textbf{gemm} matrix multiply $2mn^{2}$}
    \State $R \gets I$

    \While{$cond(G) > 100.0$}
      \State $U \Sigma V^{T} \gets svd(G)$ \Comment{\textbf{gesvd} svd $4mn^{2} - \frac{4}{3}n^{3} - \frac{2}{3}(m-n)^{3}$ } 
      \State $q r \gets qr(\sqrt{\Sigma}V^{T})$ \Comment{\textbf{geqrf} householder qr $2mn^{2} - \frac{2}{3}n^{3}$}
      \State $R \gets r R$ \Comment{\textbf{trmm} triangular matrix multiply $mn^{2}$}
      \State $Q \gets Q r^{-1}$ 

      \If {$cond(G) > 100.0$}
        \State $G = Q^{T} Q $
      \EndIf
    \EndWhile

    \State \textbf{return} $Q, R$

    \EndProcedure
  \end{algorithmic}
\end{algorithm}

\begin{table}[!ht]
  \centering
  \caption{Relevant processor specifications for the Intel i5 3570k processor}
  \label{tab:i5_3570k}

  \vspace{3mm}
  \begin{tabular}{ l | l } 
    Processor & i5 3570k \\
    Clock Rate & 3.4 GHz \\
    Cores & 4 \\
    Hyper Threading & no \\
    L1 Data Cache & 4 x 32 KB \\
    L1 Latency & 4 cycles \\
    L2 Data Cache & 4 x 256 KB \\
    L2 Latency & 11 cycles \\
    L3 Data Cache & 6 MB shared \\
    L3 Latency & 28 cycles \\
    Flops / Cycle & 8 double precision
  \end{tabular}
\end{table}

\section{Results/Discussion}

The results shown for each of the Cholesky QR factorizations are consistent with what we would expect. Tables \ref{tab:chol_qr_1000}, \ref{tab:chol_qr_2000}, and \ref{tab:chol_qr_3000} details the run times for the algorithm. To ensure that results were steady on the random matrices each test was repeated 10 times indicated by the column \textit{Iteration}. The average statistics for each matrix size can be seen in Table \ref{tab:chol_qr_avg}. We seen that the matrix $Q$ tends to be less and less orthogonal with itself as the matrix size increases. However as the matrix size increases the difference $A - QR$ remains consistent, indicating that our QR factorization scales. It is expected as discussed in Lecture 3 that the Q vectors become less orthogonal as the $A$ matrix size increases. 

As an experiment I implemented the Cholesky QR factorization in row and column major format. The difference can be seen in Table \ref{tab:chol_qr_avg} and is quite significant!

\begin{table}
  \centering

  \caption{Cholesky QR implementation run on uniform random matrices [0, 1] of size [1000, 32]}
  \vspace{3mm}

  \label{tab:chol_qr_1000}

  \begin{tabular}{c c c c c c c}
    Iteration & Real time [s] & Proc time [s] & FLOPS & MFLOPS & norm(A - QR) & norm(I - Q'Q) \\ \hline
    0 & 1.755000e-03 & 1.753000e-03 & 3274426 & 1867.898438 & 2.115893e-14 & 8.462742e-14 \\
    1 & 1.588000e-03 & 1.586000e-03 & 3252077 & 2050.489990 & 2.157851e-14 & 8.017566e-14 \\
    2 & 1.536000e-03 & 1.534000e-03 & 3261377 & 2126.060547 & 2.154383e-14 & 8.146731e-14 \\
    3 & 1.542000e-03 & 1.540000e-03 & 3271486 & 2124.341553 & 2.110507e-14 & 7.904130e-14 \\
    4 & 1.529000e-03 & 1.526999e-03 & 3250421 & 2128.632080 & 2.135736e-14 & 8.013199e-14 \\
    5 & 1.567001e-03 & 1.566000e-03 & 3311786 & 2114.805908 & 2.132766e-14 & 7.986240e-14 \\
    6 & 1.544999e-03 & 1.542998e-03 & 3282016 & 2127.035645 & 2.125873e-14 & 8.429233e-14 \\
    7 & 1.523001e-03 & 1.521999e-03 & 3255191 & 2138.758789 & 2.163561e-14 & 7.529909e-14 \\
    8 & 1.511000e-03 & 1.510000e-03 & 3232660 & 2140.834473 & 2.096920e-14 & 7.591334e-14 \\
    9 & 1.522001e-03 & 1.521001e-03 & 3245940 & 2134.082764 & 2.148730e-14 & 7.833960e-14 \\ \hline
    AVG: & 1.561800e-03 & 1.560200e-03 & 3263738 & 2095.294019 & 2.134222e-14 & 7.991504e-14 \\
  \end{tabular}
\end{table}

\begin{table}
  \centering

  \caption{Cholesky QR implementation run on uniform random matrices [0, 1] of size [2000, 32]}
  \vspace{3mm}

  \label{tab:chol_qr_2000}

  \begin{tabular}{c c c c c c c}
    Iteration & Real time [s] & Proc time [s] & FLOPS & MFLOPS & norm(A - QR) & norm(I - Q'Q) \\ \hline
    0 & 2.708001e-03 & 2.705999e-03 & 5933804 & 2192.832275 & 2.978322e-14 & 1.090657e-13 \\
    1 & 2.694000e-03 & 2.691001e-03 & 5945926 & 2209.560059 & 2.989238e-14 & 1.149029e-13 \\
    2 & 2.601001e-03 & 2.597999e-03 & 5953210 & 2291.458740 & 2.976892e-14 & 1.115480e-13 \\
    3 & 2.604999e-03 & 2.598997e-03 & 5948217 & 2288.656006 & 3.001088e-14 & 1.037338e-13 \\
    4 & 2.628002e-03 & 2.626002e-03 & 6003736 & 2286.266602 & 2.987926e-14 & 1.074047e-13 \\
    5 & 2.583999e-03 & 2.577998e-03 & 5922186 & 2297.201660 & 3.013791e-14 & 1.128357e-13 \\
    6 & 2.609000e-03 & 2.605997e-03 & 5958144 & 2286.317627 & 2.924759e-14 & 1.060727e-13 \\
    7 & 2.599001e-03 & 2.596006e-03 & 5959188 & 2295.526855 & 3.043762e-14 & 1.072767e-13 \\
    8 & 2.563000e-03 & 2.559997e-03 & 5894307 & 2302.463623 & 2.915781e-14 & 1.096877e-13 \\
    9 & 2.659999e-03 & 2.641998e-03 & 5994745 & 2269.017822 & 3.045783e-14 & 1.107151e-13 \\ \hline
    AVG: & 2.625100e-03 & 2.620199e-03 & 5951346 & 2271.930127 & 2.987734e-14 & 1.093243e-13 
  \end{tabular}
\end{table}


\begin{table}
  \centering

  \caption{Cholesky QR implementation run on uniform random matrices [0, 1] of size [3000, 32]}
  \vspace{3mm}

  \label{tab:chol_qr_3000}

  \begin{tabular}{c c c c c c c}
    Iteration & Real time [s] & Proc time [s] & FLOPS & MFLOPS & norm(A - QR) & norm(I - Q'Q) \\ \hline
    0 & 3.742002e-03 & 3.738001e-03 & 8612515 & 2304.043701 & 3.726676e-14 & 1.405381e-13 \\
    1 & 3.651001e-03 & 3.642999e-03 & 8556687 & 2348.802246 & 3.763546e-14 & 1.440986e-13 \\
    2 & 3.681995e-03 & 3.679000e-03 & 8608202 & 2339.821045 & 3.709919e-14 & 1.459294e-13 \\
    3 & 3.697000e-03 & 3.692001e-03 & 8601951 & 2329.889160 & 3.599132e-14 & 1.479825e-13 \\
    4 & 3.663003e-03 & 3.659002e-03 & 8574446 & 2343.385010 & 3.580429e-14 & 1.418014e-13 \\
    5 & 3.692001e-03 & 3.688000e-03 & 8629426 & 2339.865967 & 3.667232e-14 & 1.529708e-13 \\
    6 & 3.684998e-03 & 3.677011e-03 & 8605035 & 2340.232422 & 3.640847e-14 & 1.459143e-13 \\
    7 & 3.674999e-03 & 3.671989e-03 & 8604037 & 2343.147217 & 3.579514e-14 & 1.585664e-13 \\
    8 & 3.687009e-03 & 3.683999e-03 & 8630555 & 2342.713135 & 3.510808e-14 & 1.489244e-13 \\
    9 & 3.682002e-03 & 3.675014e-03 & 8607148 & 2342.081055 & 3.580766e-14 & 1.362737e-13 \\ \hline
    AVG: & 3.685601e-03 & 3.680702e-03 & 8603000 & 2337.398096 & 3.635887e-14 & 1.463000e-13 
  \end{tabular}
\end{table}

\begin{table}
  \centering

  \caption{Average time of Cholesky QR implementation run on uniform random matrices [0, 1]. The first group represents row major ordering while the other represents column major ordering. Notice the HUGE speed difference and scaling!}
  \vspace{3mm}

  \label{tab:chol_qr_avg}

  \begin{tabular}{c c c c c c c}
    Matrix size & Real time [s] & Proc time [s] & FLOPS & MFLOPS & norm(A - QR) & norm(I - Q'Q) \\ \hline
    1000, 32 & 1.303700e-03 & 1.300800e-03 & 710303 & 546.777570 & 2.139051e-14 & 8.313594e-14 \\
    2000, 32 & 2.115399e-03 & 2.110701e-03 & 892099 & 422.699704 & 3.014203e-14 & 1.138414e-13 \\
    3000, 32 & 2.919903e-03 & 2.914798e-03 & 1024694 & 351.532965 & 3.629157e-14 & 1.520045e-13 \\ \hline

    1000, 32 & 1.561800e-03 & 1.560200e-03 & 3263738 & 2095.294019 & 2.134222e-14 & 7.991504e-14 \\
    2000, 32 & 2.625100e-03 & 2.620199e-03 & 5951346 & 2271.930127 & 2.987734e-14 & 1.093243e-13 \\
    3000, 32 & 3.685601e-03 & 3.680702e-03 & 8603000 & 2337.398096 & 3.635887e-14 & 1.463000e-13 \\

  \end{tabular}
\end{table}



\end{document}



